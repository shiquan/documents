\documentclass[12pt]{beamer}
\usepackage{inputenc}
\usepackage{varwidth}
\usepackage{xcolor}
\definecolor{colora}{RGB}{214,76,71}
\definecolor{colorg}{RGB}{84,175,216}
\definecolor{colort}{RGB}{98,174,17}
\definecolor{colorc}{RGB}{253,205,103}
\definecolor{colorn}{RGB}{ 45,45,45}
\definecolor{colord}{RGB}{ 44,46,255}
\definecolor{colorb}{RGB}{ 14,41,89}
\definecolor{colori}{RGB}{255,44,45}

\usepackage{tikz}
\usetikzlibrary{arrows,shapes,backgrounds}
\tikzstyle{every picture}+=[remember picture]
\tikzstyle{na} = [baseline=-.5ex]
\tikzstyle{block} = [dashed, ultra thick, align=left, rounded corners, text width = \textwidth, draw=black!50, top color=white, font=\ttfamily]

\usepackage[fallback]{xeCJK}
\setCJKmainfont{STHeiti}
%\setCJKfallbackfamilyfont{rm}{SXHeiti}
%\setbeamertemplate{sections in toc}[circle]

\usecolortheme[named=colorb]{structure}
\setbeamertemplate{sections/subsections in toc}[circle]
\setbeamertemplate{itemize items}[triangle]
%\usetheme{metropolis}
\title{\huge{全基因组与目标区域重测序数据分析}}
\author{石\ 泉\\shiquan@genomics.cn\\https://github.com/shiquan}
\institute{\large{华大基因研究院}}
\date{}
\begin{document}
\frame{\titlepage}
\begin{frame}{前言}
  \tikz \node[block]{本文档为华大学院组织培训《全基因组与目标区域重测序数据
        分析》课程授课课件与相关学习文档。参考文献见各页面底部,
      请点击进去。转载请注明出处。为了保证学习讨论效果,进入本课程
      学习讨论前,请先复习以下内容。};
 \begin{columns}
 \begin{column}{0.35\textwidth}
 \begin{itemize}
 \item 生物基础 \tikz[na] \coordinate (l-i1);
 \item 测序原理 \tikz[na] \coordinate (l-i2);
 \item Linux操作基础 \tikz[na] \coordinate (l-i3);
 \end{itemize}
 \end{column}
 \begin{column}{0.65\textwidth}
 \begin{itemize}
\item \tikz \coordinate (r-i1); 中心法则(center dogma)
\item \tikz \coordinate (r-i2); DNA packaging 
\item \tikz \coordinate (r-i3); cDNA, noncoding RNA 
\item \tikz \coordinate (r-i4); 读长(reads),接头(adaptor) 
\item \tikz \coordinate (r-i5); barcoding 
\item \tikz \coordinate (r-i6); 解压压缩包 
\item \tikz \coordinate (r-i7); GNU Make,GCC 
 \end{itemize}
 \end{column}
 \end{columns}
 \begin{tikzpicture}[overlay]
 \path[->,colora,ultra thick] (l-i1) edge [bend left] (r-i1);
 \path[->,colora,ultra thick] (l-i1) edge [bend left] (r-i2);
 \path[->,colora,ultra thick] (l-i1) edge [bend left] (r-i3);
 \path[->,colorc,ultra thick] (l-i2) edge [bend left] (r-i4);
 \path[->,colorc,ultra thick] (l-i2) edge [bend left] (r-i5);
 \path[->,colorn,ultra thick] (l-i3) edge [out=0, in=-90] (r-i6);
 \path[->,colorn,ultra thick] (l-i3) edge [out=0, in=-90] (r-i7);
 \end{tikzpicture}
 \end{frame}
\begin{frame}
 \tableofcontents
 \end{frame}

\section{什么是重测序?}
\begin{frame}\frametitle{什么是重测序?}
  \end{frame}
\section{为什么进行重测序?}
\begin{frame}\frametitle{为什么进行重测序?}
  \end{frame}

\begin{frame}\frametitle{研究遗传性状(表型)的一般步骤}
  \end{frame}

\section{如何进行重测序?}
\begin{frame}\frametitle{如何进行重测序?}
  \end{frame}

\section{为什么重测序的工作中也需要组装技术?}
\begin{frame}\frametitle{为什么重测序的工作中也需要组装技术?}
  \end{frame}


\section{测序分析流程回顾}

\begin{frame}\frametitle{测序分析流程回顾如}
  \end{frame}

\section{案例分享}
\begin{frame}\frametitle{案例分享}
  \end{frame}

\section{研究思路与整体流程回顾}
\begin{frame}\frametitle{研究思路与整体流程回顾}
  \end{frame}

\begin{frame}
 
\end{frame}
\end{document}
